\newcommand{\Group}{\mathit{G}}
\newcommand{\Field}{\mathit{F}}

\newcommand{\basis}{\mathbbold{e}}
\newcommand{\Basis}{\mathcal{E}}

\DeclareMathOperator{\diag}{diag}
\DeclareMathOperator{\dist}{dist}

\newcommand{\ecc}{e}
\newcommand{\collinear}{\upuparrows}

\newcommand{\dotpdt}[2]{\left(#1; #2\right)}
\newcommand{\dotpdtv}[2]{\dotpdt{\vec{#1}}{\vec{#2}}}
\newcommand{\vecpdt}[2]{#1 \times #2}
\newcommand{\vecpdtv}[2]{\vecpdt{\vec{#1}}{\vec{#2}}}

\newcommand{\tplpdt}[3]{#1 #2 #3}
\newcommand{\tplpdtv}[3]{\tplpdt{\vec{#1}}{\vec{#2}}{\vec{#3}}}
\newcommand{\proj}[2]{\text{пр}_{#2} #1}
\newcommand{\projv}[2]{\proj{\vec{#1}}{\vec{#2}}}

\newcommand{\point}[1]{\draw[fill = black] (#1) circle (1pt);}

\tikzset{>=stealth}
\pgfplotsset{compat = 1.8}
\tikzexternalize

\tikzset{%
  external/system call = {%
    pdflatex
    -shell-escape
    -halt-on-error
    -interaction=batchmode
    -output-directory=build
    -jobname
    "\image"
    "\texsource"
  }
}

\makeatletter
\tikzoption{canvas is xy plane at z}[]{%
    \def\tikz@plane@origin{\pgfpointxyz{0}{0}{#1}}%
    \def\tikz@plane@x{\pgfpointxyz{1}{0}{#1}}%
    \def\tikz@plane@y{\pgfpointxyz{0}{1}{#1}}%
    \tikz@canvas@is@plane}
\makeatother
