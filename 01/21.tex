\subsection{%
  Системы координат. Определение. Декартовы и полярная СК.%
}

Будем говорить, что на множестве \(\Omega\) введена система координат, если
определена биекция \(S\), которая действует из \(\Omega\) в \(\RR\). Далее будем
считать \(\Omega\) геометрическим множеством (прямая, плоскость, пространство и
т.д.).

Для того, чтобы ввести СК, необходимо определить

\begin{enumerate}
\item
  Точку отсчета.

\item
  Координатную сетку.

\item
  Единичные отрезки.

\item
  Порядок осей.
\end{enumerate}

\begin{definition}
  Координатной сеткой называется семейство координатных линий/координатных
  плоскостей, таких что

  \begin{enumerate}
  \item
    Координатные линии одного семейства не пересекаются.

  \item
    Координатные линии разных семейств пересекаются в одной точке.
  \end{enumerate}
\end{definition}

\begin{remark}
  Для того, чтобы определить координаты в СК, необходимо провести кривые
  параллельные координатным линиям до их пересечения с координатными линиями
  другого семейства, которые выбраны в качестве осей (\figref{01_21_01}).
  Первая координата называется абсциссой, вторая~--- ординатой, а третья~---
  аппликатой.  
\end{remark}

\gallerydouble
  {01_21_01}{Определение координат}
  {01_21_02}{Угол между осями СК}

\begin{definition}
  Если поворот от первой оси ко второй происходит против часовой стрелки (обычно
  именно так), то СК называется правоориентированной, в противном случае СК
  называется левоориентированной.
\end{definition}

Угол между осями определяется наименьшим из положительных углов
(\figref{01_21_02}), т.е. \(\omega = \min(\alpha, \beta)\).

\begin{definition}
  Полярная система координат (\figref{01_21_03}) это СК, в которой есть
  
  \begin{enumerate}
  \item
    Полюс (начало отсчета).
  
  \item
    Полярная ось (луч \(OX\)).

  \item
    Единичный отрезок.

  \item
    Координатная сетка в виде концентрических окружностей и лучей.
  \end{enumerate}
\end{definition}

\galleryone{01_21_03}{Полярная система координат}

При этом точка \(M(\rho, \phi)\) определяется полярным радиусом \(\rho\) и
полярным углом \(\phi\) (который считается против часовой стрелки). Для того,
чтобы была биекция и одной точке соответствовала одна пара чисел, существуют
некоторые

\subheader{Ограничения}

\begin{enumerate}
\item
  \(0 \le \rho < +\infty\)

\item
  \(0 \le \phi < 2 \pi\)
\end{enumerate}
 
\begin{remark}
  Существует разные вариации ограничений угла и радиуса. Например,

  \begin{enumerate}
  \item
    \begin{enumerate}
    \item
      \(-\infty < \rho < +\infty\)

    \item
      \(0 \le \phi < \pi\)
    \end{enumerate}  

  \item
    \begin{enumerate}
      \item
        \(0 \le \rho < +\infty\)

      \item
        \(-\pi \le \phi < \pi\)
    \end{enumerate}
  \end{enumerate}
\end{remark}

Существуют некоторые виды СК.

\begin{enumerate}
\item
  Прямолинейная (декартова, ДСК)~--- координатные линии это прямые.

\item
  Прямоугольная (ДСПК)~--- угол между координатными прямыми равен \(90^\circ\).

\item
  Криволинейная~--- координатные линии это кривые.

\item
  Полярная.
\end{enumerate}
