\subsection{%
  Уравнения плоскости в пространстве.%
}

Плоскость в пространстве может быть задана с помощью

\begin{enumerate}
\item
  3ёх точек (не лежащих на одной прямой).
  
\item
  Точки и прямой, которая перпендикулярна этой плоскости.
  
\item
  Двумя пересекающимися или параллельными прямыми.
\end{enumerate}

\subheader{Общее уравнение плоскости в пространстве}

Аналогично прямой на плоскости рассматриваем нормальный вектор \(\vec{n}(A, B,
C)\), фиксированную точку \(M_0 (x_0, y_0, z_0)\) и плавающую точку \(M (x, y,
z)\).

Т.к. вектор \(\vec{n}\) перпендикулярен плоскости, то он перпендикулярен любому
вектору, лежащему в этой плоскости, значит \(\vec{n} \cdot \vec{M_0M} = 0\).
Раскрыв скалярное произведение по определению получим общее уравнение плоскости

\begin{equation*}
  A (x - x_0) + B (y - y_0) + C (z - z_0) = 0    
\end{equation*}

Если раскрыть скобки, то получится общее уравнение плоскости в пространстве

\begin{equation*}
  A x + B y + C z + D = 0
\end{equation*}

\begin{remark}
  По данному уравнению можно получить координаты вектора нормали \(\vec{n}(A, B,
  C)\).
\end{remark}

\subheader{Уравнение плоскости в пространстве через три точки.}

Рассмотрим три произвольные точки в пространстве (не лежащие на одной прямой)
\(A (x_1, y_1, z_1)\), \(B (x_2, y_2, z_2)\) и \(C (x_3, y_3, z_3)\), а также
произвольную точку \(M (x, y, z)\), лежащую в искомой плоскости.

Если точка \(M\) лежит в той же плоскости, что и точки \(A\), \(B\) и \(C\), то
векторы \(\vec{MA}\), \(\vec{MB}\) и \(\vec{MC}\) компланарны, значит их
смешанное произведение равно нулю, получаем

\begin{equation*}
  \mtxv{
    x   - x_1 & y   - y_1 & z   - z_1 \\
    x_2 - x_1 & y_2 - y_1 & z_2 - z_1 \\
    x_3 - x_1 & y_3 - y_1 & z_3 - z_1
  }
  = 0
\end{equation*}

Это называется уравнением плоскости в пространстве через три точки в форме
определителя. Если раскрыть этот определитель по определению, то получится общее
уравнение плоскости \(Ax + By + Cz + D = 0\).

\subheader{Уравнение плоскости в пространстве \quote{в отрезках}}

Если в уравнении плоскости в пространстве взять три точки, лежащие на трех осях
координат \(A (a, 0, 0)\), \(B (0, b, 0)\), \(C (0, 0, c)\), после чего
посчитать определитель, то получится уравнение плоскости \quote{в отрезках}
  
\begin{equation*}
  \frac{x}{a} + \frac{y}{b} + \frac{z}{c} = 1    
\end{equation*}

Где \(a\), \(b\), \(c\) \quote{длины} отрезков, которые плоскость отсекает от
координатных осей. Они могут быть отрицательными, это лишь говорит о том, что
отрезок лежит в отрицательной части оси.

\begin{remark}
  Получить уравнение плоскости в отрезках можно только если плоскость не
  проходит через начало координат и не параллельна координатным плоскостям.
  Иными словами плоскость должна отсекать от каждой из осей ненулевой отрезок, а
  от одного из октантов пространства тетраэдр конечного ненулевого объёма.
\end{remark}
