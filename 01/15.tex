\subsection{%
  Линейное координатное пространство. Базис, размерность.%
}

\begin{definition}
  Линейным координатным пространством называется линейное пространство числовых
  наборов одной размерности \(n\).
\end{definition}

\begin{remark}
  Базис линейного пространства определяется аналогично обычному базису
  (подробнее в \ref{sec:01-10}).
\end{remark}

\begin{definition}
  Коэффициенты разложения некого вектора по базису называется координатами в
  данном базисе.
\end{definition}

\begin{definition}
  Линейное координатное пространство называется \(n\)-мерным (обозначается
  \(L^n\), \(\dim L\) размерность пространства), если найдется линейно
  независимая система векторов размера \(n\), такая что любая система вектором
  размерности \(n + 1\) будет линейно зависимой.
\end{definition}

\begin{theorem}
  Мерность пространства \(L^n\) равна размеру базиса этого пространства.
\end{theorem}

\begin{proof}
  \(\impliedby\) Если \(\Basis = \set{\basis_i}_{i = 1}^n\) базис \(L^n\), то по
  определению базиса не найдется системы векторов размерности \(n + 1\), такой
  что она будет линейно независимой, при этом сама система \(\Basis\) линейно
  независима. Это значит, что \(\dim L = n\) по определению размерности
  пространства.
  
  \(\implies\) По определению мерности линейного координатного пространства
  найдется такая линейно независимая система \(\Basis =
  \set{\basis_i}_{i = 1}^n\), что любая система размера \(n + 1\) будет линейно
  зависимой. Покажем, что данная система является базисом линейного пространства
  \(L\). Рассмотрим систему \(\Basis \cup x\), где \(x\) произвольный элемент из
  \(L^n\).
  
  Т.к. по определению мерности линейного координатного пространства она линейно
  зависима, то существуют \(c_i\), такие что \(x c_0 + \basis_1 c_1 + \dotsc +
  \basis_n c_n = 0\). При этом \(c_0 \ne 0\) (в противном случае \(\basis_1 c_1
  + \dotsc + \basis_n c_n = 0\), а это возможно только если \(\forall c_i = 0\),
  что противоречит определению линейно зависимой системы: должна найтись
  нетривиальная нулевая комбинация). Разделим на \(c_0\) и выразим \(x\).

  \begin{equation*}
    x = -\frac{c_1}{c_0} \basis_1 - \dotsc -\frac{c_n}{c_0} \basis_n
  \end{equation*}
    
  Значит мы представили \(x\) в виде линейной комбинации системы \(\Basis\).
  Т.к. \(x\) произвольный, то любой \(x \in L^n\) может быть представлен в виде
  линейной комбинации системы \(\Basis\). Значит система \(\Basis\) базис по
  определению, причем её размер совпадает с размерностью \(L^n\).
\end{proof}
