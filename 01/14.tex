\subsection{%
  Решение однородной СЛАУ. Структура решения неоднородной СЛАУ.%
}

\begin{definition}
  Однородной СЛАУ называется СЛАУ вида \(AX = 0\).
\end{definition}

Её общее решение представимо в виде \(\mathbb{X} = c_1 X_1 + \dotsc + c_n X_n\),
где \(c_i \in \RR\) и \(\set{X_i}_{i = 1}^n\) это ФСР  данной СЛАУ. Таким
образом общее решение это линейная оболочка ФСР.

\begin{theorem}
  Однородная СЛАУ всегда совместна: у неё есть тривиальное (нулевое) решение \(X
  = 0\). Нетривиальное решение существует тогда и только тогда, когда \(\rank A
  < n\) (где \(n\) количество переменных), т.е. система столбцов линейно
  зависима.
\end{theorem}

\begin{proof}
  \(AX\) можно рассматривать как линейную комбинацию столбцов из матрицы \(A\) с
  коэффициентами \(X\). Если столбцы \(A\) линейно независимы, то у них
  существует только тривиальная нулевая линейная комбинация, т.е. уравнение \(AX
  = 0\) имеет только тривиальное решение.
\end{proof}

\begin{theorem}
  Линейная комбинация решений однородной СЛАУ также является решением этой СЛАУ.
\end{theorem}

\begin{proof}
  Пусть \(X\) решение СЛАУ \(AX = 0\), тогда \(A (\lambda X) = \lambda (AX) =
  \lambda \cdot 0 = 0\).
  
  Пусть \(X_1\) и \(X_2\) решения СЛАУ \(AX = 0\), тогда \(A (X_1 + X_2) = A X_1
  + A X_2 = 0 + 0 = 0\).
\end{proof}

\begin{definition}
  Неоднородной СЛАУ называется СЛАУ вида \(AX = B\), где \(B \ne 0\).
\end{definition}

\begin{theorem}
  Общее решение неоднородной СЛАУ \(AX = B\) представимо в виде суммы общего
  решения соответствующей однородной СЛАУ (\(AX = 0\)) и некого частного решения
  неоднородной системы.
\end{theorem}

\begin{proof}
  Пусть \(X^*\) частное решение неоднородной СЛАУ, а \(\bar{X}\) общее решение
  однородной, тогда \(A (X^* + \bar{X}) = A X^* + A \bar{X} = B + 0 = B\).
\end{proof}
