\subsection{%
  Базис. Определение, основные теоремы.%
} \label{sec:01-10}

\begin{definition}
  Базисом \(\Basis\) системы векторов \(A\) называется система, обладающая
  следующими свойствами
  
  \begin{enumerate}
  \item
    \(\Basis \subset A\)
    
  \item
    \(\Basis\) линейно независимая система.
    
  \item
    \(\forall \vec{a} \in A \given \Basis \cup \vec{a}\) линейно зависимая
    система.
  \end{enumerate}
\end{definition}

\begin{remark}
  Последнее свойство можно переформулировать так: \(\forall \vec{a} \in A \given
  \vec{a}\) разложим по системе \(\Basis\).
\end{remark}

\begin{theorem}
  Любой вектор системы разложим по базису единственным образом.
\end{theorem}

\begin{proof}
  От противного: пусть существуют два разложения вектора \(\vec{a}_p\) по базису
  \(\Basis = \set{\basis_1 \dotsc \basis_n}\). Вычтем первое из второго
  
  \begin{equation*}
    \begin{aligned}
      \vec{a}_p = \lambda_1 \basis_1 + \dotsc + \lambda_n \basis_n
    \\
      \vec{a}_p = \mu_1 \basis_1 + \dotsc + \mu_n \basis_n
    \\
      0 = (\mu_1 - \lambda_1) \basis_1 + \dotsc + (\mu_n - \lambda_n) \basis_n  
    \end{aligned}
  \end{equation*}
  
  Т.к. \(\Basis\)~--- базис, то у него есть только одна нулевая линейная
  комбинация~--- тривиальная. Значит \(\mu_i - \lambda_i = 0 \implies \mu_i =
  \lambda_i \; \forall i \in [1; n]\), т.е. разложения равны.
\end{proof}

\begin{lemma} \label{lem:extend-lin-ind}
  Если \(\bar{\Basis} \subset A\), \(\bar{\Basis}\) линейно независима, но не
  является базисом, то \(\exists \vec{a}_p \in A \setminus \bar{\Basis} \given
  \bar{\Basis} \cup \vec{a}\) является линейно независимой системой.
\end{lemma}

\begin{proof}
  От противного: если \(\forall a_p \in A \setminus \bar{\Basis} \given
  \bar{\Basis} \cup \vec{a}\) линейно зависимая система, то \(\Basis\) это базис
  \(A\) по определению. Противоречие.
\end{proof}

\begin{theorem}
  Во всякой системе, содержащий хотя бы один ненулевой вектор, можно выделить
  базис.
\end{theorem}

\begin{proof}
  Пусть у нас есть пустая подсистема.

  \begin{enumerate}  
  \item
    Добавим в подсистему один ненулевой вектор, так чтобы она осталась линейно
    независимой.
  \item
    Если подсистема является базисом, то теорема верна.
  \item
    Если подсистема не является базисом, то вернёмся к шагу 1. При этом по лемме
    \ref{lem:extend-lin-ind} обязательно найдется вектор, которым можно
    дополнить подсистему.
  \end{enumerate} 
\end{proof}
