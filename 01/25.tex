\subsection{%
  Уравнения прямой на плоскости.%
}

Прямая на плоскости может быть задана с помощью

\begin{enumerate}
\item
  Двух точек.

\item
  Точки и параллельной прямой (не проходящей через эту точку).

\item
  Точки и перпендикулярной прямой.  
\end{enumerate}

\subheader{Общее уравнение прямой на плоскости}

\begin{twocolumns}
  Имеем (\figref{01_25_01})

  \begin{enumerate}
  \item
    \(\vec{n} (A, B)\)~--- нормальный вектор, который перпендикулярен искомой
    прямой.

  \item
    \(M_0 (x_0, y_0)\)~--- фиксированная точка, через которую надо провести
    прямую.

  \item
    \(M (x, y)\)~--- плавающая точка на искомой прямой.
  \end{enumerate}

  Введём обозначения

  \begin{enumerate}
  \item
    \(\vec{r} = \vec{OM}\)

  \item
    \(\vec{r_0} = \vec{OM_0}\)

  \item
    \(\vec{M_0M} = \vec{r} - \vec{r_0} = (x - x_0, y - y_0)\)
  \end{enumerate}
\end{twocolumns}

Т.к. \(\vec{n}\) перпендикулярен искомой прямой, а \(\vec{M_0M}\) лежит на ней,
то их скалярное произведение равно нулю

\begin{equation*}
  \dotpdtv{n}{M_0M} = A (x - x_0) + B (y - y_0) = 0
\end{equation*}

Это уравнение называется (неприведенным?) общим уравнением прямой на плоскости.
Если раскрыть скобки и обозначить все числа буквой \(C\), то получится общее
уравнение прямой.

\begin{equation*}
  A x + B y + C = 0
\end{equation*}

\begin{remark}
  По данному уравнению можно получить координаты вектора нормали
  \(\vec{n}(A, B)\).
\end{remark}

\begin{remark}
  Форма задания прямой на плоскости вида \(\dotpdt{(\vec{r} -
  \vec{r_0})}{\vec{n}} = 0\) называется нормальной векторной формой.  
\end{remark}

\begin{remark}
  Уравнение вида \(k A x + k B y + k C = 0 \; (k \ne 0)\) будет задавать ту же
  прямую на плоскости.
\end{remark}

\begin{theorem}
  Общее уравнение прямой \(A x + B y + C = 0\) определяет одну и только одну
  прямую на плоскости (при условии \(A^2 + B^2 > 0\)).
\end{theorem}

\begin{proof}
  Пусть уравнение \(A x + B y + C = 0\) имеет решение \((x_0, y_0)\), тогда
  \(A x_0 + B y_0 + C = 0\). Вычтем это уравнение из исходного, получим
  \(A (x - x_0) + B (y - y_0) = 0\). Рассмотрим прямую \(l\), содержащую точку
  \((x_0, y_0)\). Она перпендикулярна вектору \((A, B)\). Как было показано
  ранее, точка и вектор нормали однозначно определяют прямую.
\end{proof}

\gallerydouble
  {01_25_01}{Общее уравнение прямой}
  {01_25_02}{Параметрическое уравнение прямой}

\subheader{Параметрическое уравнение прямой на плоскости}

\begin{twocolumns}
  Имеем (\figref{01_25_02})

  \begin{enumerate}
  \item
    \(\vec{s}(m, n)\)~--- вектор по направлению.

  \item
    \(M_0 (x_0, y_0)\)~--- фиксированная точка, через которую надо провести
    прямую.

  \item
    \(M (x, y)\)~--- плавающая точка на искомой прямой.
  \end{enumerate}

  Введём обозначения:

  \begin{enumerate}
  \item
    \(\vec{r} = \vec{OM}\)

  \item
    \(\vec{r_0} = \vec{OM_0}\)

  \item
    \(\vec{M_0M} = \vec{r} - \vec{r_0} = (x - x_0, y - y_0)\)
  \end{enumerate}
\end{twocolumns}

Т.к. \(\vec{M_0M}\) лежит на искомой прямой, а \(\vec{s}\) ему коллинеарен, то
\(\vec{M_0M} = t \vec{s}\), где \(t\)~--- это параметр. Получаем векторное
уравнение, которое можно записать в виде системы.

\begin{equation*}
  \begin{cases}
    x - x_0 = t m \\
    y - y_0 = t n
  \end{cases}
  \iff
  \begin{cases}
    x = x_0 + t m \\
    y = y_0 + t n
\end{cases}
\end{equation*}

Это уравнение называется параметрическим уравнением прямой на плоскости.

\begin{remark}
  По данному уравнению можно получить координаты вектора по направлению
  \(\vec{s}(m, n)\) и точку на прямой~--- \(M_0 (x_0, y_0)\).
\end{remark}

\subheader{Каноническое уравнение прямой на плоскости}

В параметрически заданном уравнении прямой на плоскости выразим параметр \(t\) в
обоих уравнениях и приравняем результаты. Получим каноническое уравнение прямой
на плоскости.

\begin{equation*}
  \begin{cases}
    x = x_0 + tm \\
    y = y_0 + tn
  \end{cases}
  \iff
  \begin{cases}
    \frac{x - x_0}{m} = t \\
    \frac{y - y_0}{n} = t
  \end{cases}
  \iff
  \frac{x - x_0}{m} = \frac{y - y_0}{n}
\end{equation*}

\begin{remark}
  \(m\) и \(n\) могут равняться нулю, в этом случае это рассматривается не как
  деление на ноль, а как условная запись, отображающая структуру прямой.
\end{remark}

\gallerydouble
  {01_25_03}{Уравнение прямой через две точки}
  {01_25_04}{Уравнение прямой \quote{в отрезках}}

\subheader{Уравнение прямой на плоскости через две точки}

Две точки на плоскости можно рассмотреть как точку и вектор по направлению
искомой прямой (\figref{01_25_03}). Таким образом задача построения прямой через
две точки сводится к задаче построения прямой через точку коллинеарно вектору. В
итоге получится следующая формула

\begin{equation*}
  \frac{x - x_1}{x_2 - x_1} = \frac{y - y_1}{y_2 - y_1}
\end{equation*}

\subheader{Уравнение прямой на плоскости \quote{в отрезках}}

Если в уравнении прямой на плоскости через две точки рассмотреть не произвольные
точки плоскости, а точки, лежащие на координатных осях (\figref{01_25_04}), то
его можно свести к виду

\begin{equation*}
  \frac{x}{a} + \frac{y}{b} = 1
\end{equation*}

\begin{remark}
  Получить уравнение прямой на плоскости в отрезках можно только если прямая не
  проходит через начало координат и не параллельна координатным осям. Иными
  словами прямая должна отсекать от каждой из осей ненулевой отрезок, а от одной
  из четвертей на плоскости - треугольник конечной ненулевой площади.
\end{remark}
