\subsection{%
  Произведения векторов и их приложения.%
}

Подробнее про скалярное произведение можно прочитать в \ref{sec:01-20}.

\begin{definition} \label{def:vec-pdt-geo}
  \(\vecpdtv{a}{b}\) называется векторным произведением векторов \(\vec{a}\) и
  \(\vec{b}\) (\figref{01_23_01}), если 

  \begin{enumerate}
  \item
    \(\abs{\vec{c}} = \abs{\vec{a}} \cdot \abs{\vec{b}} \cdot \sin \phi\), где
    \(\phi\) это геометрический угол между \(\vec{a}\) и \(\vec{b}\).

  \item
    \(\vec{c} \perp \vec{a}\) и \(\vec{c} \perp \vec{b}\)

  \item
    Тройка \(\vec{a}, \vec{b}, \vec{c}\) (порядок важен) правая
    (см.~\ref{rem:right-triple}).

  \item
    Если один векторов ноль, то результат ноль, т.к. угол между произвольным
    вектором и нулевым вектором не определен.
  \end{enumerate}
\end{definition}

Определение \ref{def:vec-pdt-geo} это геометрическое определение векторного
произведения, но существует также и алгебраическое определение.

\begin{definition}
  Векторным произведением векторов \(\vec{a}\) и \(\vec{b}\) называется

  \begin{equation*}
    \vecpdtv{a}{b} = \mtxv{
      \vec{i} & \vec{j} & \vec{k} \\
      a_x     & a_y     & a_z     \\
      b_x     & b_y     & b_z
    } 
  \end{equation*}

  где \((\vec{i}, \vec{j}, \vec{k})\) координаты вектора \(\vec{c} =
  \vecpdtv{a}{b}\).
\end{definition}

Векторное произведение обозначается как \(\vecpdtv{a}{b}\) или \([\vec{a};
\vec{b}]\).

\gallerydouble
  {01_23_01}{Векторное произведение}
  {01_23_02}{Правая тройка}

\begin{remark} \label{rem:right-triple}
  Изображенная на \figref{01_23_02} тройка называется правой (аналогично, если
  \(c\) направлен вниз, то левой). Аналогия с правилом буравчика: если
  \quote{идти} от \(\vec{a}\) к \(\vec{b}\), то \(\vec{c}\) должен оказаться с
  той же стороны, что и большой палец руки.
\end{remark}

\begin{remark}
  Векторное произведение некоммутативно \(\vecpdtv{a}{b} = -(\vecpdtv{b}{a})\)
  (это следует из свойств определителя матрицы), но дистрибутивность выполняется

  \begin{equation*}
    \begin{aligned}
      \vecpdt{(\vec{a} + \vec{b})}{\vec{c}} = \vecpdtv{a}{c} + \vecpdtv{b}{c}
    \\
      \vecpdt{\vec{a}}{(\vec{b} + \vec{c})} = \vecpdtv{a}{b} + \vecpdtv{a}{c}
    \end{aligned}
  \end{equation*}
\end{remark}

\begin{remark}
  Если векторное произведение равно нулю, то векторы коллинеарны (нулевой вектор
  коллинеарен любому вектору).
\end{remark}

\begin{definition}
  Смешанное произведение это комбинация векторного и скалярного произведений
  (обозначается \(\tplpdtv{a}{b}{c}\)).

  \begin{equation*}
    \tplpdtv{a}{b}{c}
    = \dotpdt{(\vecpdtv{a}{b})}{\vec{c}}
    = \mtxv{
      a_x & a_y & a_z \\
      b_x & b_y & b_z \\
      c_x & c_y & c_z
    }
  \end{equation*}
\end{definition}

\begin{remark}
  При вычислении смешанного произведения через векторное и скалярное, можно
  пользоваться тем фактом, что порядок их вычисления неважен. Таким образом
  можно вычислить сначала скалярное произведение, а потом векторное (или
  наоборот).  
\end{remark}

\begin{remark}
  По свойству определителя можно определенным образом (но не произвольно!)
  переставлять переменные в смешанном произведении.

  \begin{equation*}
    \tplpdtv{a}{b}{c} = \tplpdtv{c}{a}{b} = \tplpdtv{b}{c}{a}
  \end{equation*}

  Однако (также по свойству определителя)

  \begin{equation*}
    \tplpdtv{a}{b}{c} = -\tplpdtv{a}{c}{b}
  \end{equation*}
\end{remark}

\begin{remark}
  Если смешанное произведение равно нулю, то векторы компланарны (либо среди них
  есть нулевой вектор).
\end{remark}

\begin{theorem}
  Площадь параллелограмма равна модулю векторного произведения векторов, на
  которых он построен (\figref{01_23_03}).

  \begin{equation*}
    S = \abs{\vecpdtv{a}{b}}
  \end{equation*}
\end{theorem}

\begin{proof}
  Воспользуемся известной формулой площади параллелограмма \(S = \abs{\vec{a}}
  \abs{\vec{b}} \sin \alpha\). По геометрическому определению векторного
  произведения векторов получаем, что \(S = \abs{\vecpdtv{a}{b}}\).
\end{proof}
 
\begin{remark}
  Площадь треугольника построенного на тех же векторах, будет равна половине
  модуля векторного произведения как половина площади параллелограмма.

  \begin{equation*}
    S_{\triangle} = \frac{1}{2} \abs{\vecpdtv{a}{b}}
  \end{equation*}
\end{remark}

\gallerydouble
  {01_23_03}{Площадь параллелограмма}
  {01_23_04}{Объем параллелепипеда}

\begin{theorem}
  Объём параллелепипеда равен модулю смешанного произведения векторов, на
  которых он построен (\figref{01_23_04}).

  \begin{equation*}
    V = \abs{\tplpdtv{a}{b}{c}}
  \end{equation*}
\end{theorem}

\begin{proof}
  По известной формуле \(V = S_{\text{осн}} \cdot h\). Вычислим площадь
  основания с помощью векторного произведения \(S_{\text{осн}} =
  \abs{\vecpdtv{a}{b}}\). Заметим, что \(h\) это ортогональная проекция
  \(\vec{c}\) на векторное произведение векторов \(\vec{a}\) и \(\vec{b}\).
  Вычислим её по определению проекции.

  \begin{equation*}
    h = \frac{\dotpdt{(\vecpdtv{a}{b})}{\vec{c}}}{\abs{\vecpdtv{a}{b}}}
  \end{equation*}

  Подставим полученные значения в исходную формулу.

  \begin{equation*}
    V
    = \abs{\vecpdtv{a}{b}}
      \cdot
      \frac{\dotpdt{(\vecpdtv{a}{b})}{\vec{c}}}{\abs{\vecpdtv{a}{b}}}
    = \dotpdt{(\vecpdtv{a}{b})}{\vec{c}}
    = \abs{\tplpdtv{a}{b}{c}}
  \end{equation*}
\end{proof}

\begin{remark}
  Объём тетраэдра построенного на тех же векторах, будет равен одной шестой
  модуля векторного произведения.

  \begin{equation*}
    V_{\triangle} = \frac{1}{6} \abs{\tplpdtv{a}{b}{c}}
  \end{equation*}

  Эта формула выводится на основании двух соображений.

  \begin{enumerate}
  \item
    Объём пирамиды, построенной на тех же векторах в три раза меньше (известный
    геометрический факт).
  \item
    Основание тетраэдра вдвое меньше основания пирамиды.
  \end{enumerate}

  Таким образом получаем множитель \(\frac{1}{6}\).
\end{remark}
