\subsection{%
  Поле комплексных чисел.%
}

\begin{definition}
  Множество это совокупность элементов с общим свойством.
\end{definition}

\begin{definition}
  Множество с введённой на нём операцией называется алгебраической структурой.
\end{definition}

\begin{definition}
  Группой называется алгебраическая структура, имеющая следующие свойства
  
  \begin{enumerate}
  \item
    Замкнутость \(\forall a, b \in \Group \given a + b \in \Group\).
    
  \item
    Ассоциативность \(\forall a, b, c \in \Group \given (a + b) + c = a + (b +
    c)\).
    
  \item
    Наличие нейтрального элемента \(\exists \theta \in \Group \given \forall a
    \in \Group \given a + \theta = a\).
    
  \item
    Наличие обратного элемента \(\forall a \in \Group \given \exists a' \in
    \Group \given a + a' = \theta\).
  \end{enumerate}
\end{definition}

\begin{remark}
  В определении выше под знаком \quote{\(+\)} подразумевается любая операция.
  Группы для операции \quote{сложение} называются аддитивными. Группы для
  операции \quote{умножение} называются мультипликативными.
\end{remark}

\begin{definition}
  Если группа обладает свойством коммутативности \(\forall a, b \in \Group
  \given a + b = b + a\), то она называется абелевой (коммутативной) группой.
\end{definition}

\begin{definition}
  Кольцо это коммутативная аддитивная группа, в которой
  
  \begin{enumerate}
  \item
    Определено умножение.
    
  \item
    Относительно этого умножения выполняется дистрибутивность \(a + b \cdot c =
    a \cdot c + b \cdot c\) и \(c \cdot (a + b) = c \cdot a + c \cdot b\) (т.к.
    коммутативность для умножения не гарантирована).
  \end{enumerate}
\end{definition}

\begin{definition}
  Если кольцо обладает свойством коммутативности относительно умножения
  \(\forall a, b \in \Group \given a \cdot b = b \cdot a\), то оно называется
  коммутативным кольцом.
\end{definition}

\begin{definition}
  Поле это коммутативное ассоциативное кольцо, в котором
  
  \begin{enumerate}
  \item
    Есть нейтральный элемент по умножению \(\exists \theta \given \forall a \in
    \Field \given a \cdot \theta = a\).
    
  \item
    Для любого ненулевого элемента существует обратный элемент по умножению
    \(\forall a \in \Field \given \exists a^{-1} \given a \cdot a^{-1} =
    \theta\).
  \end{enumerate}
\end{definition}

\begin{definition}
  Множество комплексных чисел определено как множество упорядоченных пар
  вещественных чисел \(\CC = \set{ \pair{x, y} \given x, y \in \RR }\).
\end{definition}

Пусть \(z_1, z_2 \in \CC, \lambda \in \RR\) тогда

\begin{enumerate}
\item
  Сумма определена как \(z_1 + z_2 = \pair{x_1 + x_2, y_1 + y_2}\).
  
\item
  Произведение определено как \(z_1 \cdot z_2 = \pair{x_1 x_2 - y_1 y_2, x_1 y_2
  + y_1 x_2}\).
  
\item
  Умножение на число определено как \(\lambda z_1 = z_1 \lambda = \pair{\lambda
  x_1, \lambda y_1}\).
\end{enumerate}

\begin{remark}
  Проверим необходимые условия поля для комплексных чисел.
  
  \begin{enumerate}
  \item
    Замкнутость \(z_1 + z_2 = \pair{x_1 + x_2, y_1 + y_2} \in \CC\).
    
  \item
    Ассоциативность \(z_1 + (z_2 + z_3) = (z_1 + z_2) + z_3 = \pair{x_1 + x_2 +
    x_3, y_1 + y_2 + y_3}\) в силу ассоциативности сложения.
    
  \item
    Наличие нейтрального элемента по сложению \(z + \pair{0, 0} = z\)
    нейтральный элемент по сложению \(\pair{0, 0}\).
    
  \item
    Наличие обратного элемента по сложению \(\pair{x, y} + \pair{-x, -y} =
    \pair{0, 0}\).
    
  \item
    Дистрибутивность для умножения
    
    \begin{enumerate}
    \item
      \((z_1 + z_2) \cdot z_3 = \pair{ (x_1 + x_2) x_3 - (y_1 + y_2) y_3, (x_1 +
      x_2) y_3 + (y_1 + y_2) x_3 }\)
      
    \item
      \(z_1 \cdot z_3 + z_2 \cdot z_3 = \pair{ x_1 x_3 - y_1 y_3 + x_2 x_3 - y_2
      y_3, x_1 y_3 + y_1 x_3 + x_2 y_3 + y_2 x_3 }\)
    \end{enumerate}

    Значит \((z_1 + z_2) \cdot z_3 = z_1 \cdot z_3 + z_2 \cdot z_3\). Т.к.
    умножение комплексных чисел коммутативно, то достаточно проверить одну
    дистрибутивность.
    
  \item
    Наличие нейтрального элемента по умножению \(z \cdot \pair{1, 0} = \pair{x,
    y}\), нейтральный элемент по умножению \(\pair{1, 0}\).
    
  \item
    Наличие обратного элемента по умножению для любого ненулевого элемента. Для
    комплексного числа \(z = \pair{x, y}\) обратным будет являться комплексное
    число \(\display{z' = \pair{\frac{x}{x^2 + y^2}, \frac{-y}{x^2 +
    y^2}}}\) Причем т.к. \(z \ne 0\) этот элемент будет существовать. Проверим,
    что это действительно обратный элемент.
    
    \begin{equation*}
      z \cdot z'
      = \pair{x, y} \cdot \pair{
        \frac{x}{x^2 + y^2},
        \frac{-y}{x^2 + y^2}
      }
      = \pair{
        \frac{x^2}{x^2 + y^2} + \frac{y^2}{x^2 + y^2},
        \frac{-yx}{x^2 + y^2} + \frac{yx}{x^2 + y^2}
      }
      = \pair{1, 0}
    \end{equation*}

    Все верно, т.к. получился нейтральный элемент по умножению.
  \end{enumerate}
\end{remark}

\begin{definition}
  Алгебраической формой комплексного числа называется форма \(z = x + yi\), в
  которой
  
  \begin{enumerate}
  \item
    \(i^2 = -1\)~--- мнимая единица.
    
  \item
    \(x\)~--- действительная часть комплексного числа.
    
  \item
    \(y\)~--- мнимая часть комплексного числа.
    
  \item
    Под операцией \quote{\(+\)} подразумевается не сложение (нельзя сложить
    действительную и мнимую части), а просто указание на то, что это единое
    целое.
  \end{enumerate}
\end{definition}

\begin{remark}
  Комплексные числа, у которых действительная часть равна нулю называют чисто
  мнимыми. Комплексные числа, у которых мнимая часть равна нулю являются
  обычными вещественными числами.
\end{remark}

\begin{remark}
  Покажем, что \(i^2 = -1\).

  \begin{equation*}
    i^2
    = \pair{0, 1}^2
    = \pair{0, 1} \cdot \pair{0, 1}
    = \pair{0^2 - 1^2, 0 \cdot 1 + 1 \cdot 0}
    = \pair{-1, 0}
  \end{equation*}
\end{remark}

\begin{remark}
  Комплексные числа можно изображать в декартовой системе координат
  (\figref{01_01_01}).
  
  \begin{enumerate}
  \item
    Ось \textit{Re} будет отвечать за действительную часть комплексного числа.
    
  \item
    Ось \textit{Im} будет отвечать за мнимую часть комплексного числа.
  \end{enumerate}
\end{remark}

\galleryone{01_01_01}{Изображение чисел на комплексной плоскости}
