\subsection{%
  Теоремы о линейно зависимых и независимых системах векторов.%
}

\begin{theorem}
  Надсистема линейно зависимой системы линейно зависима.
\end{theorem}

\begin{proof}
  Рассмотрим исходную систему, т.к. она линейно зависима, то найдутся такие
  коэффициенты \(\lambda_1 \dotsc \lambda_n\), при которых линейная комбинация
  векторов этой системы будет равна нулю.
  
  При это среди этих коэффициентов точно будет хотя бы один ненулевой (в
  противном случае линейная комбинация была бы тривиальной, а значит система
  была бы линейно независимой). Далее рассмотрим надсистему, возьмём векторы из
  исходной системы с коэффициентами \(\lambda_1 \dotsc \lambda_n\), а остальные
  векторы с коэффициентом \(0\).
  
  Мы получили нулевую линейную комбинацию, в которой точно найдется ненулевой
  коэффициент, из чего следует, что надсистема линейно зависима.
\end{proof}

\begin{theorem}
  Подсистема линейно независимой системы векторов линейно независима.
\end{theorem}

\begin{proof}
  От противного: если подсистема была бы линейно зависимой, то по теореме о
  линейной зависимости надсистемы векторов исходная система (которая является
  надсистемой для рассматриваемой подсистемы) также была бы линейно зависимой.
  
  Получили противоречие, а значит подсистема линейно независима.
\end{proof}
